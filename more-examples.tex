
\chapter{More examples}

\section{Equations}

Here's an equation,
\begin{equation}
E=mc^2.\label{eq:einstein}
\end{equation}

\noindent I can reference that easily in the text: \autoref{eq:einstein}. It's even a hyperlink. How nice. 

\section{Opening and closing quotes}

Unlike modern word processors, you need to specify in \LaTeX{} which quote mark to print. To get an opening quote you use a backtick and the regular apostrophe for a closing quote. Double them up for speech. ``This isn't so hard after all''. One just needs to `get used' to it. 

One should never use an apostrophe for plurals. Nope, not even for abbreviations, e.g. in the 1990s, people bought CDs from Virgin Megastores. 

In the \textbf{extremely} rare cases where it's unclear, match it with an opening quote if you must. I got three `A's for my AS Levels. 

\section{Tables}

Tables are joyous fun. The \verb+tabular+ environment is the most common, although it's rather old fashioned and wrangling it into doing what you want can be arcane. Happily, tablesgenerator.com can produce tables from a visual editor or paste from word.

A few things to help you unlearn bad table habits:

\begin{itemize}
    \item You should not use vertical lines in tables. Seriously -- this is an awful 1990s era default from Microsoft Word which has hung around and never gone away. 
    \item the booktabs package can make prettier tables (vertical lines are intentionally banned) - select this option in tablesgenerator. I have included the package for you
    \item You should use tables for comparing numerical data and not as a way of laying out content or paragraph text
\end{itemize}

\begin{table}
    \centering
    \begin{tabular}{lccc}
        \toprule
        \textbf{Feature} & \textbf{Liked (\%)} & \textbf{Disliked (\%)} & \textbf{Didn't know (\%)}  \\ \midrule
        Vertical lines & 0 & 90 & 10 \\
        Using Word & 40 & 40 & 20 \\
        \bottomrule
    \end{tabular}
    \caption{Made up percentages of participants that liked random features}
    \label{tab:sample}
\end{table}

\noindent \autoref{tab:sample} shows a simple table made by hand by yours truly. Note that the column separator is \& which means you must always escape that character if you want to use it in text.



