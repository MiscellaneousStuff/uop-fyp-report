\chapter{Introduction} \label{chap:intro}

\section{Project Background}

The problem I will investigate is how to classify electrical signals from a person's facial
muscles captured using non-invasive surface electromyography (sEMG) into text without
them speaking (i.e. silently articulated speech).I believe this project is worth working on
because non-invasive silently articulated speech devices are a human computer interface
which offer unique benefits which other methods do not. A non-exhaustive list of examples
is provided below:

\emph{Privacy of conversation:} Typical speech recognition systems have users broadcast
what they're saying to the environment (e.g. issuing commands to an Amazon Alexa
device) and therefore privacy is not maintained. A silent speech device does not require the
user to say anything aloud, rather they're only required to slightly move their facial muscles
to mouth out what they would like to say.

\emph{Eavesdropping:} Voice interfaces are always listening in on conversations, when not
desired, only to be visibly activated later on by a specific trigger-word such as ‘Ok Google'.
A silent speech device could avoid this entirely by providing a physical mechanism for the
user to start recording what they would like to say instead.

\emph{Attention Requiring:} Existing voice interaction devices have low usability as they
require the user to pay full attention to what they're saying and how the device is
responding. Also, proximity to the device is required as using any voice system from far
away reduces it's effectiveness. Silent speech systems would avoid this issue as they
would be directly recording what the user is intending to say directly from the electrodes
which are attached to the skin of the user.

The above examples will benefit all users of silent speech systems. However, these
systems could also uniquely benefit users who have medical issues which make regular
speech difficult such as people who suffer from Multiple Sclerosis with Dysphonia. There
has already been promising research into using silent speech devices to help individuals
with varying levels of speech impairments which can benefit from silent speech systems,
and further research into healthy or unhealthy people will benefit all future users of silent
speech systems.
(\cite{pmlr-v116-kapur20a})

\section{Project Objectives}

The primary objectives for this project are threefold:

\begin{itemize}
    \item \textit{Improve SOTA Approach:}
    \item \textit{EMG Data Acquisition:}
    \item \textit{Open-Source Release:}
\end{itemize}

\section{Report Structure}