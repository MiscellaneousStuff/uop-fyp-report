\chapter{Introduction} \label{chap:intro}

\section{Project Background}

The problem I will investigate is how to classify electrical signals from a person's facial
muscles captured using non-invasive surface electromyography (sEMG) into text without
them speaking (i.e. silently articulated speech). I believe this project is worth working on
because non-invasive silently articulated speech devices are a human computer interface
which offer unique benefits which other methods do not. A non-exhaustive list of examples
is provided below:

\emph{Privacy of conversation:} Typical speech recognition systems have users broadcast
what they're saying to the environment (e.g. issuing commands to an Amazon Alexa
device) and therefore privacy is not maintained. A silent speech device does not require the
user to say anything aloud, rather they're only required to slightly move their facial muscles
to mouth out what they would like to say.

\emph{Eavesdropping:} Voice interfaces are always listening in on conversations, when not
desired, only to be visibly activated later on by a specific trigger-word such as 'Ok Google'.
A silent speech device could avoid this entirely by providing a physical mechanism for the
user to start recording what they would like to say instead.

\emph{Attention Requiring:} Existing voice interaction devices have low usability as they
require the user to pay full attention to what they're saying and how the device is
responding. Also, proximity to the device is required as using any voice system from far
away reduces it's effectiveness. Silent speech systems would avoid this issue as they
would be directly recording what the user is intending to say directly from the electrodes
which are attached to the skin of the user.

The above examples will benefit all users of silent speech systems. However, these
systems could also uniquely benefit users who have medical issues which make regular
speech difficult such as people who suffer from Multiple Sclerosis with Dysphonia. There
has already been promising research into using silent speech devices to help individuals
with varying levels of speech impairments which can benefit from silent speech systems,
and further research into healthy or unhealthy people will benefit all future users of silent
speech systems.
(\cite{pmlr-v116-kapur20a})

\section{Project Aims}

The overall aim of this project is to contribute to open-source research concerning
sEMG based silent speech systems. Originally this meant creating an improved model
for sEMG silent speech and using an EMG dataset acquired from myself and other
participants and then open-sourcing the new model and dataset.

However after I decided to not pursue the data acquisition part of my project,
the aim of my project was instead, still focused on the research and development
of novel methods to improve sEMG silent speech systems.

\section{Project Objectives}

As eluded to in the project initiation document, there are three objectives
for this project, with one primary objective, one supplementary objective
and a another objective which was mainly reliant on the primary objective.

The first and most important objective to achieving the aim of this project
was to discover methods within the EMG silent speech literature and intuitions
gained from analysing an open-source silent speech dataset to determine how
to improve on existing methods.
Formally, my objective here was to improve
on existing methods by outperforming a state-of-the-art model on a task
based on at least one evaluation metric. Before I commenced this project,
I found the following two metrics were of particular interest within the
silent speech domain: inference time (milliseconds between the model
receiving an input and producing an output) and WER (word-error rate,
a raw measure of the accuracy of a model) and training dataset sizes
(number of hours of data required to train a system).

The second objective for my project was to purchase an EMG data acquisition
device and then create a new EMG silent speech dataset from at least
myself, and if possible from willing participants. The purpose of this
was to address the gap I found when trying to find open-source EMG silent
speech datasets.

The third objective of this project was to release an EMG silent speech
dataset to the public which could further be used by other researchers
to accelerate progress in the silent speech research domain. The other
part of this objective was to release the improved approach for silent
speech into the public domain as an open-source repository on GitHub.
This would make it available to any future researchers to build upon.