\chapter{Approaches} \label{chap:approaches}

This chapter will describe the main dataset used for this research, why it was chosen,
the main EMG signals of interest from the dataset, methods of feature selection and
visualisation and the machine learning approaches used to transcribe the EMG data
into a text.

\section{Dataset}

\subsection{Dataset Selection and Justification}

The dataset which is used throughout this research project is the open-source
surface electromyography silent speech (sEMG silent speech) dataset released
by David Gaddy along with his paper, Digital Voicing of Silent Speech
(\cite{gaddy2020digital}).
The paper describes a novel method of transcribing aligned silent speech data
directly into speech features along with the largest open-source sEMG silent
speech dataset.

This dataset was chosen for this research as it is the largest, quality open
source sEMG silent speech dataset.

\subsection{Feature Selection}

For this research there were two primary ways of selecting features. The first method
was to use the same feature processing methods described in the original
(\cite{gaddy2020digital}) paper. The second method was to use a convolutional
neural network (CNN) architecture to automatically learn features from the
dataset, in an end-to-end manner.