\iffalse
Mark Scheme:
- Extensive Research
- Understanding of complex subject matter
- Identifies flaws, gaps or inconsistencies in extant knowledge
\fi

\iffalse

Mu-Law compression:
- compresses audio signal into discrete bins whilst preserving dynamic range

Current sEMG Silent Speech Text Classification Research:
- https://dspace.mit.edu/bitstream/handle/1721.1/123121/1128187233-MIT.pdf?sequence=1&isAllowed=y

\fi

\chapter{Literature Review} \label{chap:lit-review}

This section of the report identifies the early research directions which were considered,
along with explaining basic terms in the EMG silent speech domain. The goal of this section
is to briefly familiarise the reader with the silent speech domain and silent speech
interfaces in general. Then the two key approaches to silent speech systems are explored,
silent speech to audio and silent speech to text.

\section{Key Terms}

A list of silent speech and speech recognition domain specific key terms are provided
to familiarise the reader with certain concepts to help the reader digest this report.

\begin{itemize}
  \item \emph{Word Error Rate (WER):}
  Metric used in speech recognition systems which accounts for differences in length of
  a recognised word sequence and a reference word sequence. WER is derived from
  Levenshtein distance  which is a string metric for measuring the difference between two
  sequences. The Levenshtein distance between two words is essentially the minimum number of
  single-character edits (including insertions, deletions or substitutions) required to
  change one word into another. Another term for it is the "edit distance".
  \[
    WER
    = \dfrac{S + D + I}{N}
    = \dfrac{S + D + I}{S + D + C}
  \]
  Where each term means:\\
  S: Number of substitutions\\
  D: Number of deletions\\
  I: Number of insertions\\
  C: Number of correct words\\
  N: Number of words in the reference (N = S+D+C)
  (\cite{1966SPhD...10..707L})
  \item \emph{Mel-Scale:}
  Perceptual scale of pitches judged by listeners to be equal in distance from one another.
  \[ m = 2595 \log_{10} \left(1 + \dfrac{f}{700}\right) \]
  (\cite{mel_scale_formula})
  \item \emph{Silent Speech Interface (SSI):}
  "A silent speech interface (SSI) is a system enabling speech communication to take
  place where an audible acoustic signal is unavailable. By acquiring sensor data from
  elements of the human speech production process, an SSI produces a digital representation
  of speech which can be synthesized directly, interpreted as data or routed into a communications
  network" (\cite{ssi_definition}).
  \item \emph{Surface Electromyography (sEMG):}
  Non-invasive, computer-based technique that records the electrical impulses placed on the surface
  of the skin overlying the nerve at rest (i.e. static) and during activity (i.e. dynamic).
  \item \emph{Automatic Speech Recognition (ASR):}
  The process of creating a text transcription of speech, known as a word sequence,
  in which the focus is on the shape of the speech wave (\cite{asr_definition}).
\end{itemize}

\section{Machine Learning Background}

\subsection{David Gaddy and Dan Klein}

David Gaddy and Dan Klein have published two papers relating to the transduction of
silent speech. These two papers are of particular interested for silent speech research
as they describe a novel method of transducing silent speech into speech features.
This makes it easier to produce speech synthesis and text classifiction systems
for silent speech research. Also, the original paper also contains an accompanying
open-source dataset which is comparable to smaller speech recognition datasets
such as LJSpeech (\cite{ljspeech17}). This makes the two papers and their open-source
dataset critical to the success of this dissertation, as only on in the project it
was decided to not pursue the EMG data acquisition.

\subsubsection{Original Paper}

The paper digitally voices silent speech from silently mouthed words which are
converted to audible speech based on EMG sensor measurements which capture muscle movements.
The novel contribution of this paper is  
a method of transducing speech features directly from EMG recordings
of a person's face during silently articulated speech. The main innovative approach
suggested in this paper is to transfer audio targets from vocalised EMG recordings
to silent EMG recordings by learning the alignment between silent EMG recordings
and ground truth audio features.

For each utterance in the dataset, silent EMG and vocalised EMG recordings are recorded
in two different sessions. However, only the vocalised EMG recording also has an associated
ground truth audio file (as silent speech involves not actually producing noise when
speaking). Then the silent EMG recording is aligned with the vocalised EMG recording.
This makes it possible to directly transduce silent EMG recordings into speech.


\iffalse
Other papers have used this approach to FILL REST OF THIS WITH REFERENCED TO SPEECH AND TO TEXT
FROM DIGITAL VOICING, MIT ALTER EGO 2018 SUMMARISED AND 60 PG VERSION AND OTHER SOURCES.
\fi

(\cite{gaddy2020digital})

\subsubsection{An Improved Model for Voicing Silent Speech}

Evolution of Digital Voicing paper which improves model performance compared to previous work.
Model learns its own input features directly from EMG signals instead of hand-crafted features in prior work.
New model uses convolutional layers to extract features from the signals.
Transformer-encoder layers used to propagate signals across longer distances instead of bi-directional LSTM layers.
Additional signal introduced during learning by introducing auxillary task of predicting phoneme labels in addition to predicting speech audio features.
On open vocabulary intelligibility evaluation, the new model improves the absolute WER by 25.8\%.
(\cite{gaddy2021improved})

Establishes new SOTA performance via 3 main improvements:

\begin{itemize}
  \item \emph{Replace LSTMs with Transformer-Encoder:}
  Model bi-directional triple stacked 1024-unit LSTM layer replaced with 6 transformer-encoder
  layer to predict EEG features found using CNN over 1 second to predict speech along with
  phoneme loss signal.
  Ablation shows WER go from 42.2\% to 45.2\% when this is removed.
  \item \emph{Introduce Phoneme Loss:}
  Introduced auxillary phoneme prediction task to as additional 
  output of self-attention transformer-encoder. This signal is used to improve the performance
  of the network by learning to predict the phoneme of the speech which is being detected and
  also regularises training.
  Ablation shows WER go from 42.2\% to 51.7\% when this is removed.
  \item \emph{Replace hand-designed features with convolution features:}
  Uses CNN block based on ResNet to extract features from EMG signals which are end-to-end
  trainable instead of using hand-crafted features to improve the representational power of the network.
  Ablation shows WER go from 42.2\% to 46\% when this is removed.
\end{itemize}

(\cite{gaddy2021improved})

\iffalse

\section{Silent Speech Interfaces for Speech Restoration: A Review}

\section{Proposed Directions for Research}

\subsection{Text Classification}

\subsection{Data Augmentation}

It might be possible to use GANs (Generative Adversarial Networks) or
VAEs (Variational Auto Encoders) to create a model which can generate
more data samples.

One possible data augmentation method for sEMG based silent speech is to
reverse a SOTA transduction model. Existing transduction models for
EMG to speech use either EMG features or raw EMG data and transduce
it into speech features (typically mel spectrograms or MFCCs). However,
it may be possible to predict the EMG data from the mel spectrograms.

\subsection{Improving Existing Models}

\subsubsection{Model Improvements}

Recently there has been more research into silent speech models which
can directly produce text from muscle activity. One paper introduces
a sequence-to-sequence voice reconstruction model and training regime
called \textit{SSRNet} to address what they call the sEMG2V problem
in a tonal language.

combines
multiple methods into one to achieve strong performance in synthesizing
speech in Mandarin Chinese from EMG signals. The authors of this paper
propose a complex architecture, where the model is based on the
FastSpeech (\cite{fastspeech}) neural network architecture.

The FastSpeech
neural network architecture is to convert text, it it's phonemic
representation into mel spectrograms. Compared to other speech
synthesis models, FastSpeech includes novel neural network components.
These include a Feed-Forward Transformer, FFT Block, Length Regulator
and Duration Predictor. The purpose of these neural network blocks
is to improve the abiltiy of the model to correctly predict the duration
of a phoneme. This is particularly important because for text to speech
models, it is difficult to determine the alignment of the inputs to
the outputs as the only information which is provided is text, which
does not contain timing information.

\section{Connectionist Temporal Classification (CTC)}

For any speech recognition task, a model must know the alignment
between the input (e.g. audio, EMG features, etc.) and the target
transcription. On the surface, this makes training any speech
recognition model difficult.

Without having the alignments between the input and the transcription,
simpler approaches aren't available to us, such as mapping a single
character to a fixed number of inputs. This is because people's rates
of speech vary, regardless of the input modality (e.g. audio, EMG features,
etc). Another option is to hand-label the alignments between the input
and the text transcriptions. This would produce a performant model, however,
hand-labeling the alignments is time consuming, especially for larger datasets.

Connectionist Temporal Classification (CTC) is a method to train a model
without knowing the alignment between the input and the output and is especially
well suited to labeled datasets such as speech recognition.
It was originally proposed in a 2006 paper (\cite{ctc_original}) and specifically
dealt with training recurrent neural networks (RNN) (\cite{rnn_fundamentals}).
\fi