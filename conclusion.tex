\chapter{Conclusion} \label{chap:conclusion}

Overall this project was highly successful in achieving it's primary objective,
which was to find a method of improving either silent speech classification
or transduction based on at least one evaluation metric. Instead a method is
proposed which achieves the same WER as a prior state-of-the-art (SOTA)
system, 68\% for the Digital Voicing approach (\cite{gaddy2020digital})
and 68.26\% for the proposed approach from this project while requiring
115 times less training data to train the ASR component of the system
and removing the vocoder component which would been required for the
previous approach.

Secondly, the objective to open-source the method and dataset used for
this approach was also successful with the successful
speech recognition approach
\footnote{https://github.com/MiscellaneousStuff/semg-asr}
and less successful EMG synthesis approach
\footnote{https://github.com/MiscellaneousStuff/semg\_silent\_speech\_py}
both being open-sourced before the submission of this project.

However the additional objective to create an open-source sEMG silent
speech dataset and release it to the public was unfortunately not achieved.
This is unfortunate as there are very few available silent speech datasets
available and the creation and release of such a dataset would of been
highly beneficial for researchers.

In conclusion, this project ultimately provides a method which reduces
the required data and machine learning system components for silent
speech recognition