\section{Fine Tuning Speech Recognition on Model Predictions}

The intuition behind fine tuning a speech recognition model on the
predicted mel spectrograms predicted by the transduction model comes
from the literature for speech recogntion. Typical speech recognition models
suffer from a loss of accuracy when they are only trained on clean audio data
and then evaluated on a noisy dataset (\cite{DS2_original}).

This issue is also reflected in the transduction task. The state-of-the-art
transduction model within the Improved Voicing (\cite{gaddy2021improved}) paper
will always produce audio outputs with artefacts because of the final vocoder layer.

This becomes problematic if you want to use the transduction model as a basis for a
speech recognition system. This is because most pre-trained speech recognition systems
are trained on clean audio datasets.

\subsection{Closed Vocabulary Dataset}

The closed vocabulary dataset is the first dataset used to determine how much better
a speech recognition model can be improved by training on the predicted mel spectrograms
from an already trained transduction model.

The WER of the closed vocabulary ASR model trained only on the training dataset
of the closed vocab dataset is 37\%. This means that we would not expect the WER
of the model when it's evaluated on transduced examples to perform better than
the ground truth word-error rate.

% (Desc.) UoP: FYP: SEMG ASR (Finetune DS2 on Transducer Outputs #8)
{\small\begin{center}
\captionof{table}{DeepSpeech2 Closed Vocab Finetuned Results}
\begin{tabular} {  l  c  c  }
\hline
\textbf{Dataset} & \textbf{CER} & \textbf{WER} \\
\hline
Ground Truth (GT) & 87.10 & 100.50 \\
Voiced & 51.27 & 87.33 \\
Silent & 38.80 & 78.10 \\
Silent, Voiced & \textbf{35.26} & 75.33 \\
\hline
Silent, Voiced, GT & 35.72 & \textbf{70.83} \\
\hline
\end{tabular}
\end{center}}

The above results do not use phonemeic prediction and only use greedy decoding
in the decoder. DeepSpeech2 is also not the SOTA model for speech recognition
which also reduces the performance. This means that the best WER of 70.83 could
be improved upon a lot more.

For the silent, voiced, GT condition, the model is pre-trained on the
ground truth model and and then fine-tuned on the predictions of the
silent and voiced utterances.

\subsection{Open Vocabulary (Parallel-Only) Dataset}

The next dataset used is the parallel voiced and silent EMG data. This is used because
it's smaller than using the entire open vocabulary condition dataset in one go which
makes training the individual ASR models and the transduction model faster.

The WER of the ASR model trained only on the
training dataset of the open vocabulary parallel voiced audio is 64.47\%.

% (Desc.) UoP: FYP: SEMG ASR (Finetune DS2 on Transducer Outputs #8)
{\small\begin{center}
\captionof{table}{DeepSpeech2 Open Vocabulary Parallel Finetuned Results}
\begin{tabular} {  l  c  c  }
\hline
\textbf{Dataset} & \textbf{CER} & \textbf{WER} \\
\hline
Ground Truth (GT) & 66.70 & 110.49 \\
Voiced & 156.29 & 100.00 \\
Silent & 44.21 & 84.19 \\
Silent, Voiced & 42.11 & 85.30 \\
Voiced, GT & 43.06 & 84.65 \\
Silent, GT & 41.24 & 83.97 \\
\hline
Silent, Voiced, GT & \textbf{41.02} & \textbf{81.69} \\
\hline
\end{tabular}
\end{center}}

Due to the results of the closed vocabulary training condition and time constraints,
training on the voiced portion of the open vocabulary paralell mel spectrogram
predictions wasn't conducted.

There is one interesting difference in this training run compared to the closed
vocabulary dataset. Training on the silent EMG and voiced EMG conditions together
reduces performance. This may be because training on both together was better
on training on just the silent EMG signals for the closed vocabulary dataset
because the silent EMG dataset only contained 400 data samples so doubling
the dataset to 800 data samples, even though the vocalised samples are sub-optimal,
is better. However, here the silent EMG dataset is comprised of 2,778 data samples
which means that the difference between the silent EMG and voiced EMG predicted
mel spectrograms is reducing the performance of the ASR model more than having a
larger overall dataset is beneficial.